%----------------------------------------------------------------------------------------
%    PACKAGES AND THEMES
%----------------------------------------------------------------------------------------

\documentclass[aspectratio=169,xcolor=dvipsnames]{beamer}
\usetheme{SimplePlus}
\usepackage{tikz}  % 用于绘图
\usepackage{amsmath}  % 更好的数学公式支持
\usepackage{amssymb}  % 数学符号
\usetikzlibrary{shapes, arrows, positioning, calc}  % TikZ库
\usepackage{bm}
\usepackage{xcolor}
\usetikzlibrary{decorations.pathm, patterns}
\usepackage{hyperref}
\usepackage{graphicx} % Allows including images
\usepackage{booktabs, tabularray} % Allows the use of \toprule, \midrule and \bottomrule in tables
\graphicspath{{figures/}}

%----------------------------------------------------------------------------------------
%    TITLE PAGE
%----------------------------------------------------------------------------------------
\title{曲梁结构免自锁有限元无网格法\\及其在屈曲分析中的应用}

\author{汇报人:何祖彦}

% 新增指导老师信息
\newcommand{\advisorname}{指导老师:吴俊超} % 请在此处填入指导老师的姓名

% 自定义标题页模板,包含指导老师信息和校徽
\setbeamertemplate{title page}{%
 % 在左上角添加文字
    \begin{flushleft}
        \vspace{0.5em}
        \hspace{0.5em}
        \begin{minipage}[t]{0.8\textwidth}
            \raggedright
            \usebeamerfont{institute}
            \color{DarkBlue}
            \textbf{土木工程学院2024级开题汇报}\\
        \end{minipage}
    \end{flushleft}
    \vspace{0.1em} % 调整顶部间距为logo留出空间
    
    % 在顶部添加华侨大学校徽
    \begin{center}
        \includegraphics[width=3cm]{hqu2.png} % 调整校徽大小
        \vspace{0.3em}
    \end{center}
    
    \begingroup
    \centering
    % ------------------------
    \begin{beamercolorbox}[sep=1pt,center]{title}
        \usebeamerfont{title}\inserttitle\par%
        \ifx\insertsubtitle\@empty%
        \else%
        \vskip0.25em%
        {\usebeamerfont{subtitle}\usebeamercolor[fg]{subtitle}\insertsubtitle\par}%
        \fi%
    \end{beamercolorbox}%
    \vskip1em\par
    % ------------------------
    \begin{beamercolorbox}[sep=3pt,center]{author}
        \usebeamerfont{author}\insertauthor
    \end{beamercolorbox}
    % 添加指导老师信息,与作者名字保持1pt间距
    \vskip1pt
    \begin{beamercolorbox}[sep=5pt,center]{author}
        \usebeamerfont{author}\advisorname
    \end{beamercolorbox}
    \vskip0.5em
    % ------------------------
    \begin{beamercolorbox}[sep=3pt,center]{date}
        \usebeamerfont{date}\insertdate
    \end{beamercolorbox}\vskip0.5em
    % ------------------------
    {\usebeamercolor[fg]{titlegraphic}\inserttitlegraphic\par}
    \endgroup
    \vfill
}

\date{2025年12月14日} % Date, can be changed to a custom date

%----------------------------------------------------------------------------------------
%    PRESENTATION SLIDES
%----------------------------------------------------------------------------------------

\begin{document}

\begin{frame}
    % Print the title page as the first slide
    \titlepage
\end{frame}

\begin{frame}{目录}
    % Throughout your presentation, if you choose to use \section{} and \subsection{} commands, these will automatically be printed on this slide as an overview of your presentation
    \tableofcontents
\end{frame}

\section{研究背景}

% 在您的 LaTeX 文档中合适位置添加以下代码
\begin{frame}{研究背景}

\begin{center}
% 并排插入两张图片
\begin{minipage}[t]{0.48\linewidth}
    \centering
    \includegraphics[width=\linewidth]{16.jpg} \\[0.5em]
\end{minipage}
\hfill
\begin{minipage}[t]{0.48\linewidth}
    \centering
    \includegraphics[width=\linewidth]{17.png} \\[0.5em]
\end{minipage}
\end{center}

\vspace{1em}

\begin{center}
\textbf{主要研究对象为考虑横向剪切应变的曲梁结构}
\end{center}

\end{frame}

\begin{frame}{研究背景}
  \vspace{-0.5cm} % 减少顶部空白

\begin{columns}[onlytextwidth,T]
    % 左侧列:公式和文字
    \begin{column}{0.55\textwidth}
        \vspace{0.8cm}
        
        % 位移部分
        \textbf{曲率:}
        \begin{align*}
            \kappa=\theta,_{s}  
        \end{align*}
        
        \vspace{1cm}
        
        % 应变部分
        \textbf{应变:}
        \begin{align*}
            \gamma &= w,_{s} - \theta + \frac{u}{R} \\
            \varepsilon &= u,_{s} - \frac{w}{R} \\
        \end{align*}
        
        \vspace{0.2cm}

    \end{column}
    
    % 右侧列:图形
    \begin{column}{0.45\textwidth}
        \centering
        \vspace{0.5cm}
        
        \begin{center}
            \includegraphics[width=3.5cm]{1.png} % 调整校徽大小
            \vspace{0.3em}
        \end{center}

         \begin{center}
            \includegraphics[width=3.5cm]{12.png} % 调整校徽大小
            \vspace{0.3em}
        \end{center}
    \end{column}
\end{columns}

\end{frame}

\begin{frame}{研究背景}

% 主公式区域
\begin{align*}
& \text{应力-应变关系:} \\
& \begin{bmatrix}
\sigma_{11} \\
\sigma_{22} \\
\sigma_{12}
\end{bmatrix}
=
\frac{E}{1-\nu^{2}}
\begin{bmatrix}
1 & \nu & 0 \\
\nu & 1 & 0 \\
0 & 0 & \frac{(1-2\nu)}{2}
\end{bmatrix}
\begin{bmatrix}
\varepsilon_{11} \\
\varepsilon_{22} \\
2\varepsilon_{12}
\end{bmatrix},
\qquad
\begin{bmatrix}
\sigma_{13} \\
\sigma_{23}
\end{bmatrix}
= 2G
\begin{bmatrix}
\varepsilon_{13} \\
\varepsilon_{13}
\end{bmatrix} \\[1em]
& M_{\alpha\beta} = \int_{0}^{h} x_{3}\sigma_{\alpha\beta}dx_{3}
\quad
Q_{\alpha} = k\int_{0}^{h}\sigma_{\alpha 3}dx_{3} \\[1em]

\text{弯矩:} 
\begin{align*}
     M = EI \frac{d\theta}{ds} \\
\end{align*}
\text{剪力:} 
\begin{align*}
     Q = \kappa GA \left( \frac{dw}{ds} - \theta \right)
\end{align*}
\end{align*}
\end{frame}

\begin{frame}{研究背景}

% 伽辽金弱形式
\begin{block}{伽辽金弱形式:}
\vspace{0.5em}
\begin{multline*}
D^b \int_{\Omega} \delta \bm{\kappa}^T \bm{D} \bm{\kappa} \, d\Omega 
+ D^s \int_{\Omega} \delta \bm{\gamma}^T \bm{\gamma} \, d\Omega 
+ \int_{\Gamma^h} \delta \bm{\theta}^T \overline{\bm{M}} \, d\Gamma 
+ \int_{\Gamma^h} \delta w^T \overline{\bm{Q}} \, d\Gamma 
- \int_{\Omega} \delta w \, q \, d\Omega = 0
\end{multline*}
\end{block}

\vspace{0.8em}

% 向量和矩陣定義
\begin{block}{}

\begin{align*}
\bm{\kappa} &= 
\begin{bmatrix}
\varphi_{1,1} \\[0.5em]
\varphi_{2,2} \\[0.5em]
\varphi_{1,2} + \varphi_{2,1}
\end{bmatrix} &
\bm{\gamma} &=
\begin{bmatrix}
w_{,1} - \varphi_1 \\[0.5em]
w_{,2} - \varphi_2
\end{bmatrix} &
\bm{D} &=
\begin{bmatrix}
1 & \nu & 0 \\
\nu & 1 & 0 \\
0 & 0 \dfrac{1 - 2\nu}{2}
\end{bmatrix}
\end{align*}
\end{block}

\vspace{0.8em}

\end{frame}

\begin{frame}{研究背景}
    \vspace{2em}
    % 抗弯刚度和抗剪刚度
\begin{block}{抗弯刚度和抗剪刚度}
\vspace{0em}
\begin{align*}
D^b = \frac{E h^3}{12(1 - \nu^2)} \quad
D^s = k G h
\end{align*}
\end{block}

\vspace{2em}
% 紅色警示框
\begin{center}
\begin{tikzpicture}
    \node[draw=red, line width=2pt, rounded corners=8pt, fill=red!8, 
          inner xsep=1.2em, inner ysep=0.8em, text width=0.9\textwidth, align=center] (warning) {
        \Large\bfseries\textcolor{red}
        h → 0 時$D^s$抗剪刚度过大导致剪切自由度受到约束,引起自锁
    };
    
    % 锯齿状装饰边框
    \draw[decorate, decoration={zigzag, segment length=4pt, amplitude=3pt, pre=lineto, pre length=5pt, post=lineto, post length=5pt}, 
          line width=1.5pt, red] (warning.south west) -- (warning.south east);
    \draw[decorate, decoration={zigzag, segment length=4pt, amplitude=3pt, pre=lineto, pre length=5pt, post=lineto, post length=5pt}, 
          line width=1.5pt, red] (warning.north west) -- (warning.north east);
\end{tikzpicture}
\end{center}
% 调整位置
\vspace{2em}
\end{frame}

\begin{frame}{研究背景}

         \begin{center}
            \includegraphics[width=10cm]{20.jpg} % 调整校徽大小
            \vspace{0.3em}
        \end{center}
\begin{block}{屈曲理论}
\vspace{0em}
\begin{align*}
P_{cr} = \frac{\pi^2 E I}{K L^2}
\end{align*}
\end{block}

\begin{center}
    \begin{tikzpicture}[
        redbox/.style={draw=red, rectangle, minimum width=4.5cm, minimum height=1.2cm, align=center, line width=1.8pt, rounded corners=5pt, font=\large\bfseries, fill=red!5},]
        

        \node[redbox] (solution) {结构失稳的临界载荷};
    \end{tikzpicture}
\end{center}
\end{frame}

\begin{frame}{研究背景}
  \vspace{-0.5cm} % 减少顶部空白

\begin{columns}[onlytextwidth,T]
    % 左侧列:公式和文字
    \begin{column}{0.4\textwidth}
        \vspace{0.8cm}
        \textbf{混合离散分析方法} \\
        \vspace{0.3cm}
        % 位移部分
        \textbf{挠度离散:}
        \begin{align*}
            u^h(x) = \sum_{I=1}^{n_u} N^u_{I}(x) d_{i} \\  
        \end{align*}
        
        % 应变部分
        \textbf{应力离散:}
        \begin{align*}
            Q^h(x) = \sum_{I=1}^{n_q} M^q_{I}(x) q_{j} \\
        \end{align*}
        
        \vspace{0.2cm}

    \end{column}
    
    % 右侧列:图形
    \begin{column}{0.6\textwidth}
        \centering
        \vspace{1cm}
        \begin{center}
            \includegraphics[width=7.6cm]{22.png} % 调整校徽大小
            \vspace{0.3em}
        \end{center}
    \end{column}
\end{columns}

\end{frame}

\begin{frame}{研究背景}

% 带问号的双向箭头
\begin{center}
\begin{tikzpicture}[baseline]
    % 双向箭头
    \draw[<->, line width=3.5pt, black] (0.5,0) -- (5.5,0);
    
    % 箭头上的问号
    \node[circle, draw=red, line width=1.2pt, fill=white, inner sep=2pt] at (3,0) {\textcolor{red}{\Huge\bfseries ?}};
    
    % 左侧标注
    \node[anchor=east] at (0,0) {\large\bfseries $\inf_{q \in Q} \sup_{w \in H} \frac{b(w,q)}{\|w\| \|q\|} \geq \beta > 0$};
    
    % 右侧标注
    \node[anchor=west] at (6,0) {\large\bfseries $\dim(w):\dim(q)$};

\end{tikzpicture}
\end{center}

\vspace{2em}

% 底部结论框
\begin{center}
    \begin{tikzpicture}[
        redbox/.style={draw=red, rectangle, minimum width=10cm, minimum height=1.2cm, align=center, line width=1.8pt, rounded corners=5pt, font=\large\bfseries, fill=red!5},]
        

        \node[redbox] (solution) {LBB稳定性条件和自由度比例之间的关系不明确};
    \end{tikzpicture}
\end{center}

\end{frame}

\begin{frame}{研究背景}
  \vspace{-0.5cm} % 减少顶部空白
\begin{columns}[onlytextwidth,T]
    % 左侧列:公式和文字
    \begin{column}{0.35\textwidth}
         \centering
        \vspace{0.3cm}
        
        \begin{center}
            \includegraphics[width=4cm]{12.png} % 调整校徽大小
            \vspace{0.3em}
        \end{center}

         \begin{center}
            \includegraphics[width=3.5cm]{15.png} % 调整校徽大小
            \vspace{0.3em}
        \end{center}
    \end{column}
    
    % 右侧列:图形
    \begin{column}{0.65\textwidth}
        \vspace{0.5cm}
       \begin{tblr}{
    width = \textwidth,
    colspec = {X[1, c] X[1, c] X[1, c] X[1, c]},
    row{1} = {bg=MediumBlue!20, fg=black, font=\bfseries},
    row{2-Z} = {abovesep=0.5pt, belowsep=0.5pt},
    row{16} = {font=\normalsize},
    hlines = {1pt, black},
    vlines = {},
}

$\frac{h}{R}$ & $|\frac{u^f}{u^c}-1|$ & $|\frac{v^f}{v^c}-1|$ & $|\frac{\theta^f}{\theta^c}-1|$ \\

1/5    & 0.35\%  & 1.59\%  & 2.09\%  \\
1/10   & 2.57\%  & 3.24\%  & 3.57\%  \\
1/20   & 9.08\%  & 9.30\%  & 9.04\%  \\
1/30   & 17.83\% & 17.82\% & 16.80\% \\
1/40   & 27.48\% & 27.28\% & 25.54\% \\
1/50   & 36.90\% & 36.58\% & 34.26\% \\
1/100  & 69.27\% & 68.83\% & 65.96\% \\
1/200  & 89.75\% & 89.54\% & 88.06\% \\
1/300  & 95.13\% & 95.02\% & 94.24\% \\
1/400  & 97.19\% & 97.13\% & 96.66\% \\
1/500  & 98.18\% & 98.14\% & 97.83\% \\
1/1000 & 99.54\% & 99.53\% & 99.45\% \\
\end{tblr}
    \end{column}
\end{columns}

\vspace{0.5cm}
\end{frame}

\begin{frame}{研究背景}
      \vspace{-0.5cm} % 减少顶部空白
\begin{columns}[onlytextwidth,T]
    % 左侧列:公式和文字
    \begin{column}{0.35\textwidth}
         \centering
        \vspace{0.3cm}
        
        \begin{center}
            \includegraphics[width=4cm]{12.png} % 调整校徽大小
            \vspace{0.3em}
        \end{center}
        
         \begin{center}
            \includegraphics[width=3.5cm]{14.png} % 调整校徽大小
            \vspace{0.3em}
        \end{center}
    \end{column}
    
    % 右侧列:图形
    \begin{column}{0.65\textwidth}
        \vspace{0.3cm}
        \begin{center}
            \includegraphics[width=8.5cm]{13.png} % 调整校徽大小
            \vspace{0.3em}
        \end{center}
    \end{column}
\end{columns}
\end{frame}

\begin{frame}{研究背景}

\begin{columns}[onlytextwidth,T]

% 左侧栏:问题与解决方案
\begin{column}{0.5\textwidth}
\begin{flushleft}  % 左对齐
\vspace{-0.5em}  % 调整顶部间距

% 问题标题
\begin{center}
{\Large\bfseries 问题:}
\end{center}

\vspace{0.8em}

% 问题内容
\begin{itemize}
    \setlength{\itemsep}{0.8em}  % 增加项目间距
    \item[\textbf{1.}] 消除自锁的最优自由度比例。
    \item[\textbf{2.}] 如何任意调整约束自由度比例。
\end{itemize}
\end{flushleft}
\end{column}

% 右侧栏:美化流程图
\begin{column}{0.5\textwidth}
\centering
\vspace{0.5em}

% 美化流程图
\begin{tikzpicture}[
    node distance=1.5cm,
    box/.style={draw, rectangle, minimum width=3.2cm, minimum height=1cm, align=center, line width=1.2pt, rounded corners=4pt, font=\large\bfseries},
    redbox/.style={draw=red, rectangle, minimum width=4.5cm, minimum height=1.2cm, align=center, line width=1.8pt, rounded corners=5pt, font=\large\bfseries, fill=red!5},
    plus/.style={draw, circle, minimum size=1.2cm, line width=1.2pt, font=\Large\bfseries},
    arrow/.style={->, >=stealth, line width=1.2pt}
]

% 有限元框
\node[box, fill=blue!10] (fem) at (0,0) {有限元};

% 加号
\node[plus, fill=gray!10] (plus) at (0,-1.5) {+};

% 无网格框
\node[box, fill=green!10] (meshfree) at (0,-3) {无网格};


% 解决方案框
\node[redbox] (solution) at (0,-5.5) {内禀最优约束比的\\有限元无网格混合分析方法};

% 连接线
\draw[arrow, red, line width=1.8pt, shorten >=0.1cm, shorten <=0.1cm] (0,-3.7) -- (solution);

\end{tikzpicture}
\end{column}

\end{columns}

\end{frame}

\section{研究方案}

\begin{frame}{研究方案}
    \vspace{-0.5em}
% 并排显示两张图片
\begin{columns}[onlytextwidth,c]

% 左栏:有限元法
\begin{column}{0.48\textwidth}
\begin{center}
% 插入有限元法图片
\includegraphics[width=\textwidth]{18.png}

% 有限元法描述
\begin{tikzpicture}
    \node[draw=blue!50!black, fill=blue!10, thick, rounded corners=5pt, 
          inner seppt=8pt, text width=0.9\textwidth, align=center] {
        \Large\bfseries\textcolor{blue!50!black}{有限元法}
        
        \vspace{0.3em}
        
        \begin{minipage}{0.85\textwidth}
        \centering
        \begin{itemize}
            \setlength{\itemsep}{0.1em}
            \item 基于网格划分
            \item 形函数不连续
            \item 单元边界断裂
            \item 需网格生成
        \end{itemize}
        \end{minipage}
    };
\end{tikzpicture}
\end{center}
\end{column}

\vspace{0.5em}
% 右栏:无网格法
\begin{column}{0.48\textwidth}
\begin{center}
% 插入无网格法图片
\includegraphics[width=\textwidth]{25.png}


% 无网格法描述
\begin{tikzpicture}
    \node[draw=green!black, fill=green!10, thick, rounded corners=5pt, 
          inner sep=8pt, text width=0.9\textwidth, align=center] {
        \Large\bfseries\textcolor{green!50!black}{无网格法}
        
        \vspace{0.3em}
        
        \begin{minipage}{0.85\textwidth}
        \centering
        \begin{itemize}
            \setlength{\itemsep}{0.1em}
            \item 无需网格划分
            \item 形函数全域连续
            \item 光滑高阶逼近
            \item 适应性强
        \end{itemize}
        \end{minipage}
    };
\end{tikzpicture}
\end{center}
\end{column}
\end{columns}
\end{frame}

\begin{frame}{研究方案}
\vspace{0.3cm}
\textbf{有限元无网格混合离散}
\begin{columns}[onlytextwidth,T]
\begin{column}{0.55\textwidth}
        \vspace{0.3cm}
        \begin{center}
            \includegraphics[width=8.5cm]{27.png} % 调整校徽大小
            \vspace{0.3em}
        \end{center}
    \end{column}

\begin{column}{0.45\textwidth}
        \vspace{-0.4cm}
        \begin{center}
            \includegraphics[width=4cm]{26.png} % 调整校徽大小
            \vspace{0.3em}
        \end{center}
    \end{column}
\end{columns}
\end{frame}


\section{进度安排}

\begin{frame}{研究进度安排}
    \vspace{0.2cm} % 减少顶部空白
% 使用 tabularray 包创建美观的表格
\begin{tblr}{
    width = \textwidth,
    colspec = {X[1.02, l] X[2.3, l] X[0.8, c]},
    row{1} = {bg=MediumBlue!20, fg=black, font=\bfseries},
    row{2-Z} = {abovesep=5pt, belowsep=5pt},
    row{6} = {font=\normalsize},
    cell{7}{1} = {c=3}{r},
    hlines = {1pt, black},
    vlines = {},
}
\textbf{时间} & \textbf{研究内容} & \textbf{进度情况} \\

2023.7-2023.12 & 验证LBB稳定性条件与自由度比例之间的关系 & 
\raisebox{-0.2ex}{\begin{tikzpicture}[baseline=(a.base)]
    \node[circle, fill=black, inner sep=3pt] (a) {};
\end{tikzpicture}} \\

2024.1-2024.2 & 进行Timoshenko曲梁的无网格有限元混合离散分析 & 
\raisebox{-0.2ex}{\begin{tikzpicture}[baseline=(a.base)]
    \node[circle, draw=black, thick, inner sep=3pt] (a) {};
\end{tikzpicture}} \\

2024.3-2024.9 & 曲梁结构免自锁在屈曲分析中的应用 & \\

2024.10-2024.12 & 整理成果并完成论文初稿撰写 & \\

2025.1-2025.3 & 进一步修改完善论文,形成并提交论文终稿 & \\

\scalebox{2}{$\bullet$}已完成 \quad \scalebox{2}{$\circ$} 正在进行 & & \\
\end{tblr}

% 在表格和图例之间添加弹性空间
\vspace*{\fill}

% 在页面底部添加2em的固定距离
\vspace{2em}
\end{frame}

\begin{frame}
    \Huge{\centerline{\textbf{请各位老师批评指正!}}}
\end{frame}

\end{document}