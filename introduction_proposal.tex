% ====================================================================
% 使用说明
% ====================================================================
% 本文档为开题报告正文模板,请将以下示例内容替换为您的实际开题报告内容。
% 
% 1. 章节标题使用方法:
%    \section{一级标题}
%    \subsection{二级标题}
%    \subsubsection{三级标题}
%
% 2. 参考文献引用方法:
%    使用 \cite{引用标签} 命令,例如:\cite{knuth1984texbook}
%    多个文献引用:\cite{knuth1984texbook,lamport1994latex}
%
% 3. 公式添加方法:
%    行内公式:$E=mc^2$
%    独立公式:
%    \begin{equation}
%        E = mc^2
%    \end{equation}
%
% 4. 列表添加方法:
%    无序列表:
%    \begin{itemize}
%        \item 第一项
%        \item 第二项
%    \end{itemize}
%    
%    有序列表:
%    \begin{enumerate}
%        \item 第一项
%        \item 第二项
%    \end{enumerate}
% ====================================================================

\section{论文选题的依据:}

随着工程结构轻量化与高性能化发展,数值仿真对精度和效率的要求日益提高。
复合材料结构分析中,各向异性导致的特殊剪切行为使传统梁理论难以满足精度需求。
Timoshenko曲梁模型通过独立描述弯曲与剪切效应,为此类材料提供精准建模基础,
其同时考虑弯曲变形和剪切变形的影响,与经典Euler-Bernoulli梁理论形成鲜明对比——后者基于平截面假定,
忽略剪切变形,仅适用于细长梁,在描述短粗梁或剪切效应显著的结构时存在明显不足。
当应用Timoshenko理论分析细长梁时,理论要求剪切变形趋近于零。
然而,有限元\cite{wang2019a}单元若采用不匹配的形函数和完全积分方案,会强制产生虚假剪切变形,
导致系统表现出过高剪切刚度,引发「剪切自锁」现象,计算位移远小于理论值,
本质是数学上的过度约束。此现象源于挠度与转角场插值阶次不匹配,当梁细长时,
虚假剪切应变能主导弯曲响应,使模型过于刚硬。无网格\cite{wang2019a}方法作为新兴数值技术,
通过移动最小二乘形函数独立插值挠度与转角,摆脱单元网格束缚,
从根本上避免基于单元离散的数值病理,尤其当基函数采用高阶多项式时效果显著。
但其计算成本较高,因形函数构建需大型支持域和矩阵求逆,效率低于传统有限元方法,
限制了大规模工程应用。发展高精度、高效率且免自锁的数值方法,成为重要研究课题。


\subsection{研究背景}

这是一个二级标题示例。可以使用行内公式,如爱因斯坦质能方程 $E=mc^2$,或者使用独立公式:

\begin{equation}
    \int_{a}^{b} f(x) \, dx = F(b) - F(a)
    \label{eq:fundamental}
\end{equation}

其中,公式\ref{eq:fundamental}表示微积分基本定理。

\subsection{研究意义}

研究意义可以分为以下几点:

\begin{itemize}
    \item 理论意义:丰富相关理论体系
    \item 实践意义:为实际应用提供指导
    \item 创新意义:提出新的研究方法
\end{itemize}

\section{文献综述}

文献综述部分需要对现有研究进行梳理和评述\cite{author2023,coauthor2022}。

\subsection{国内研究现状}

有序列表示例:

\begin{enumerate}
    \item 第一个研究方向的现状分析
    \item 第二个研究方向的现状分析
    \item 第三个研究方向的现状分析
\end{enumerate}

\subsection{国外研究现状}

国外学者在该领域也进行了大量研究,取得了显著成果。

\section{研究内容与方法}

\subsection{研究内容}

本研究的主要内容包括:

\begin{itemize}
    \item 内容一:具体描述
    \item 内容二:具体描述
    \item 内容三:具体描述
\end{itemize}

\subsection{研究方法}

采用的研究方法如下:

\begin{enumerate}
    \item 文献研究法
    \item 实验研究法
    \item 数据分析法
\end{enumerate}

\section{研究计划与预期成果}

\subsection{研究计划}

详细的研究时间安排和阶段性目标。

\begin{figure}[H]
\centering
\includegraphics[width=0.8\linewidth]{1.png} 
\caption{研究计划时间表}
\end{figure}

\subsection{预期成果}

预期将取得以下成果,并发表相关论文\cite{future2024}。
% ====================================================================
% 注意:请在 references.bib 文件中添加对应的参考文献条目
% ====================================================================