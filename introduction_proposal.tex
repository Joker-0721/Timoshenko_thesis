% ====================================================================
% 使用说明
% ====================================================================
% 本文档为开题报告正文模板,请将以下示例内容替换为您的实际开题报告内容。
% 
% 1. 章节标题使用方法:
%    \section{一级标题}
%    \subsection{二级标题}
%    \subsubsection{三级标题}
%
% 2. 参考文献引用方法:
%    使用 \cite{引用标签} 命令,例如:\cite{knuth1984texbook}
%    多个文献引用:\cite{knuth1984texbook,lamport1994latex}
%
% 3. 公式添加方法:
%    行内公式:$E=mc^2$
%    独立公式:
%    \begin{equation}
%        E = mc^2
%    \end{equation}
%
% 4. 列表添加方法:
%    无序列表:
%    \begin{itemize}
%        \item 第一项
%        \item 第二项
%    \end{itemize}
%    
%    有序列表:
%    \begin{enumerate}
%        \item 第一项
%        \item 第二项
%    \end{enumerate}
% ====================================================================

\section{论文选题的依据:}

随着工程结构轻量化与高性能化发展,数值仿真对精度和效率的要求日益提高。复合材料结构分析中,各向异性导致的特殊剪切行为使传统梁理论难以满足精度需求。Timoshenko曲梁模型通过独立描述弯曲与剪切效应,为此类材料提供精准建模基础,其同时考虑弯曲变形和剪切变形的影响,与经典Euler-Bernoulli梁理论形成鲜明对比——后者基于平截面假定,忽略剪切变形,仅适用于细长梁,在描述短粗梁或剪切效应显著的结构时存在明显不足。当应用Timoshenko理论分析细长梁时,理论要求剪切变形趋近于零。然而,有限元\cite{liu2022}单元若采用不匹配的形函数和完全积分方案,会强制产生虚假剪切变形,导致系统表现出过高剪切刚度,引发「剪切自锁」\cite{malkus1978}现象,计算位移远小于理论值,本质是数学上的过度约束。此现象源于挠度与转角场插值阶次不匹配,当梁细长时,虚假剪切应变能主导弯曲响应,使模型过于刚硬。无网格\cite{LXJZ200901002}方法作为新兴数值技术,通过移动最小二乘形函数独立插值挠度与转角,摆脱单元网格束缚,从根本上避免基于单元离散的数值病理,尤其当基函数采用高阶多项式时效果显著。但其计算成本较高,因形函数构建需大型支持域和矩阵求逆,效率低于传统有限元方法,限制了大规模工程应用。发展高精度、高效率且免自锁的数值方法,成为重要研究课题。

Timoshenko曲梁在有限元分析中常出现剪切自锁现象,这是由于横向位移 $\boldsymbol{w}$和旋转场 $\theta$的插值阶次不匹配导致的。为消除锁闭效应,减缩积分和选择性积分技术被广泛采用。Malkus\cite{malkus1978}与 Hughes\cite{hughes1977}等人证明了混合有限元方法与减缩-选择性积分位移方法的等价性,通过减少积分点放松应变约束,适用于线性和非线性问题。基于此方法Stolarski和Belytschko\cite{stolarski1982} 与de Melo等人\cite{demelo1992}针对曲单元与弯管通过局部降阶积分方案,通过减少高阶应变项的积分点规避复杂几何下的锁闭风险;Choi等人\cite{choi2023}则采用选择性降阶基函数策略减轻锁定效应。另一方面,Bouclier\cite{bouclier2012}与Cheng等人\cite{cheng1997}则提出将缩减积分与高阶技术耦合处理剪切能项,通过结合单点高斯积分和气泡函数空间,同步减少膜和剪切应变能的积分点消除锁定。然而,Bouclier\cite{bouclier2012}与Zhang等人\cite{zhang2018}指出此方法会因沙漏模式引发单元不稳定;并而在考虑高阶基函数和非均匀单元间连续性时变得相当复杂,限制其通用性;此外,Hernández和Vellojin\cite{hernandez2021}指出,相比混合约束应变方法,减缩积分方案通过减少约束方程数量缓解锁定的效果较弱。

因此,学者转向混合方法通过独立近似应力场与位移场从变分原理层面消除剪切自锁现象。Saleeb和Saleeb\cite{saleeb1987}基于Hellinger-Reissner变分原理独立插值应力和位移场,通过静态凝聚消除内部应力参数并引入约束指数准则确保无锁定性能;Afsin Saritas\cite{saritas2009}则采用Hu-Washizu三场变分原理独立插值位移、应力和应变场,通过力插值函数精确满足平衡方程以消除锁定;L. Beirão da Veiga等\cite{beiraodaveiga2012}、 L. Greco等\cite{greco2017}与Bouclier等\cite{bouclier2013}则将该范式延伸至等几何分析,利用B样条基函数高阶连续性引入独立剪切应变变量与优化应力场插值,解决高阶基函数引发的数值锁定。此外,贺立新等\cite{2009024402.nh}采用分区混合算法,在物面附近黏性主导区使用间断Galerkin有限元法,在非黏性区采用有限体积法,通过通量守恒实现平滑过渡以兼顾锁定缓解与计算效率;胡凯等\cite{LXXB202207026}则提出强-弱耦合单元微分法,在应力奇异区采用伽辽金弱形式,其余区域保留强形式单元微分法,通过结合两种方法的优势优化计算精度与效率。Baier-Saip等人\cite{baier-saip2020}指出混合插值法通过独立近似场消除剪切锁定,性能优于纯位移法。然而,因需处理多个独立场和全局矩阵求逆,导致刚度矩阵稠密化并增加计算负担;方法实现复杂,需谨慎选择离散空间以满足稳定性准则,Baier-Saip等人\cite{baier-saip2020}强调必须满足Ladyzhenskaya-Babuska-Brezzy稳定性条件,否则可能出现伪模式,而Hale和Baiz\cite{hale2012}指出混合变分公式需严格处理基函数特性以避免精度损失。

混合方法虽能从理论上消除锁定,但其计算成本高且实现复杂。为平衡精度与效率,因此,Elguedj等人\cite{elguedj2007}提出的B-bar投影体系解决剪切锁定问题;Bouclier等\cite{bouclier2012}将此扩展至任意曲率梁,耦合选择性缩减积分实现无剪切修正分析。同时,Greco等\cite{greco2017}与Zhang等\cite{zhang2018}分别采用局部/全局投影样条重建,构建膜锁定通用解决方案,通过重构应变场插值增强数值鲁棒性。此外,Patton等人\cite{patton2025}在等几何假定自然应变方法的基础上,借助NURBS高阶连续性优势,通过局部投影将应变场解耦为独立分量消除多重锁定,避免全局矩阵求逆;Prathap和Shashirekha\cite{shashirekha1993}则通过变分一致性重构膜应变场,在保持位移插值不变的前提下优化曲梁单元精度,显著降低计算复杂度。然而,Elguedj等人\cite{elguedj2007}的B-bar方法在高阶应用中需结合其他技术以完全消除锁定,而Greco等人\cite{greco2017}强调假定自然应变方法中合适采样点选取困难且易引入误差。

此外,高阶插值法和混合插值法作为替代方案也被广泛研究。Likang Li\cite{li1990}通过增加多项式次数p,而Koziey和Mirza\cite{koziey1994}以及Dawe\cite{dawe1974}则通过阶次优化策略(采用三次位移和二次旋转多项式、五次多项式)确保一致性,从而避免虚假剪切应变;Caillerie等\cite{caillerie2015}与Tai和Chan\cite{tai2016}基于此引入分层高阶形函数或内部自由度,在提升精度的同时规避锁定问题。在混合插值法方面,Saleeb和Chang\cite{saleeb1987}基于Hellinger-Reissner变分原理独立插值应力与位移场,Saritas\cite{saritas2009}则利用Hu-Washizu三场变分原理独立插值位移、应力和应变场,均通过松弛约束有效消除膜锁定。此外,Wang等\cite{wang2014}、王晓峰等\cite{LXXB201302021}、Bathe和Dvorkin\cite{bathe1985}通过引入内部节点和Hermitian/Lagrange独立插值,在精确考虑弯扭耦合效应的同时避免了横向剪切锁定。形函数方法中,陈太聪等\cite{LXXB200906019}与田荣\cite{LXXB201901027}提出构造精确形函数并设计迭代算法,使粗网格能准确捕捉屈曲载荷和失稳模态,其方法可同时用于弱/强形式离散且无需网格加密即可实现高阶收敛。然而,这些方法均存在局限,高阶插值需足够大的p值才能完全消除锁定,计算成本高且不适用于实时应用;混合插值法实现复杂且计算昂贵;形函数方法构造依赖特定问题而缺乏通用性。

上述传统有限元法基于单元建立形函数的近似方法难以在确保插值精度的同时消除剪切自锁现象。无网格法是一类不依赖单元建立形函数方法的新兴研究方向。Wang和Chen\cite{wang2004}基于MLS/RK近似和稳定共轭节点积分提出无网格曲梁公式,通过曲率平滑技术消除剪切和膜锁定,可精确再现纯弯曲模式而无寄生变形;Donning和Liu\cite{donning1998}采用基数样条函数作为形函数,结合位移基Galerkin方法在粗网格下实现高精度,并通过插值层面点对点消除锁定;Hale和Baiz等人\cite{hale2012}基于混合变分公式,融合最大熵基函数与旋转Raviart-Thomas-Nédélec单元,可直接施加狄利克雷边界条件并避免剪切锁定;Huang等人\cite{huang2021}基于CSPM法自动满足自由边界条件,适用于大变形但同样未明确针对锁定问题;Xiao和McCarthy\cite{xiao2003}与Atluri和Zhu\cite{atluri1998}利用MLPG方法结合无锁公式和移动最小二乘近似处理,通过改变控制方程变量消除锁定,无需网格离散;Li等人[45]采用对称SPH方法改进一致性误差,适用于功能梯度材料但未直接解决剪切锁定。然而,无网格方法存在本质边界条件处理困难,Donning和Liu\cite{donning1998}指出其框架中满足本质边界条件仍是重大挑战;Xiao和McCarthy\cite{xiao2003}提到因近似扩散特性导致低阶时锁定缓解效果有限。

尽管现有研究已发展出减缩积分、假定应变场、混合方法、高阶插值及无网格方法等多种技术以抑制Timoshenko曲梁分析中的剪切自锁现象,这些方法仍存在诸多未解决的根本问题,如数值稳定性与计算效率难以兼顾、实现复杂度高、通用性受限,以及缺乏对自锁机理的统一理论指导。为此,本文将以Timoshenko曲梁为研究对象,重点探讨其混合离散分析中消除剪切自锁的“最优约束比例”问题,旨在通过理论分析与数值实验明确不同离散策略下约束条件与自锁行为之间的内在关联,从而填补现有方法在理论指导层面的空白。在此基础上,本文将进一步建立一种“自锁约束比例可控的有限元与无网格混合离散分析方法”,通过有机结合有限元的结构化离散优势与无网格法的灵活近似特性,实现约束比例的可控调节,在保证计算精度的同时有效避免自锁现象,为工程实践提供一种兼具鲁棒性与适应性的新型数值解决方案。


\section{研究特色及创新之处}

1.Timoshenko曲梁混合离散分析中消除剪切自锁问题的最优约束比例。
2.自锁约束比例可控的有限元无网格混合离散分析方法。

\section{研究内容与方法}

\subsection{研究内容}

本文将针对Timoshenko曲梁剪切自锁问题,研究位移场和约束场的自由度数量比与LBB稳定性条件影响机理,
并利用最优约束比和无网格法的离散便利性,从而建立Timoshenko曲梁最优约束比的有限元无网格混合分析方法。具体内容如下:


\begin{enumerate}
    \item 建立Timoshenko曲梁有限元无网格混合离散理论框架,设计横向位移 $\boldsymbol{w}$和旋转场 $\theta$采用二次再生核无网格形函数离散;
    约束场采用有限元离散,并研究可任意调整自由度数量的无网格法节点布置方案,从而优化形函数影响域大小。
    \item 将Timoshenko曲梁最优约束比例引入有限元无网格混合离散分析方法中,
    从而建立最优约束比的免自锁有限元无网格混合离散分析方法。为复杂曲梁结构的锁紧问题提供了新的解决路径。
\end{enumerate}

\section{研究计划与预期成果}
预期将取得以下成果:

\begin{enumerate}
\item 建立Timoshenko曲梁剪切自锁与LBB稳定性条件的关系,确定剪切自锁的最优约束比例。
\item 建立曲梁最优约束比的有限元无网格混合离散分析方法,在保证计算精度的同时消除自锁现象。
\end{enumerate}

\subsection{研究计划}

\section{论文的研究进展和进度安排}
\renewcommand{\labelitemi}{$\scriptstyle\bullet$}
\begin{tblr}{
    width = 0.95\linewidth,
    colspec = {X[1.2, l] X[2.3, l] X[0.8, c]},
    row{1} = {font=\bfseries, c}, % 表头背景色
    row{2-Z} = {rowsep=8pt}, % 行间距
    row{6} = {font=\normalsize},
    cell{7}{2} = {c=2}{r},
    hline{1,2,7,8} = {1pt}, % 顶部、表头下、底部的线
    % hline{3,4,5,7} = {0.5pt, solid}, % 行之间的细线
    vline = {}, % 不显示垂直线
}
时间 & 研究内容 & 进度情况 \\
2023.7-2023.12 &  & \scalebox{2}{$\bullet$} \\
2024.1-2024.2 &  & \scalebox{2}{$\circ$} \\
2024.3-2024.9 &  &  \\
2024.10-2024.12 &  &  \\
2025.1-2025.3 &  &  \\
& \scalebox{2}{$\bullet$}已完成 \quad \scalebox{2}{$\circ$} 正在进行 & \\
\end{tblr}
\end{table}

% ====================================================================
% 注意:请在 references.bib 文件中添加对应的参考文献条目
% ====================================================================