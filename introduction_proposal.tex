% ====================================================================
% 使用说明
% ====================================================================
% 本文档为开题报告正文模板,请将以下示例内容替换为您的实际开题报告内容。
% 
% 1. 章节标题使用方法:
%    \section{一级标题}
%    \subsection{二级标题}
%    \subsubsection{三级标题}
%
% 2. 参考文献引用方法:
%    使用 \cite{引用标签} 命令,例如:\cite{knuth1984texbook}
%    多个文献引用:\cite{knuth1984texbook,lamport1994latex}
%
% 3. 公式添加方法:
%    行内公式:$E=mc^2$
%    独立公式:
%    \begin{equation}
%        E = mc^2
%    \end{equation}
%
% 4. 列表添加方法:
%    无序列表:
%    \begin{itemize}
%        \item 第一项
%        \item 第二项
%    \end{itemize}
%    
%    有序列表:
%    \begin{enumerate}
%        \item 第一项
%        \item 第二项
%    \end{enumerate}
% ====================================================================

\section{论文选题的依据:}

随着工程结构轻量化与高性能化发展,数值仿真对精度和效率的要求日益提高。
复合材料结构分析中,各向异性导致的特殊剪切行为使传统梁理论难以满足精度需求。
Timoshenko曲梁模型通过独立描述弯曲与剪切效应,为此类材料提供精准建模基础,
其同时考虑弯曲变形和剪切变形的影响,与经典Euler-Bernoulli梁理论形成鲜明对比——后者基于平截面假定,
忽略剪切变形,仅适用于细长梁,在描述短粗梁或剪切效应显著的结构时存在明显不足。
当应用Timoshenko理论分析细长梁时,理论要求剪切变形趋近于零。
然而,有限元\cite{}单元若采用不匹配的形函数和完全积分方案,会强制产生虚假剪切变形,
导致系统表现出过高剪切刚度,引发「剪切自锁」现象,计算位移远小于理论值,
本质是数学上的过度约束。此现象源于挠度与转角场插值阶次不匹配,当梁细长时,
虚假剪切应变能主导弯曲响应,使模型过于刚硬。无网格\cite{wang2019a}方法作为新兴数值技术,
通过移动最小二乘形函数独立插值挠度与转角,摆脱单元网格束缚,
从根本上避免基于单元离散的数值病理,尤其当基函数采用高阶多项式时效果显著。
但其计算成本较高,因形函数构建需大型支持域和矩阵求逆,效率低于传统有限元方法,
限制了大规模工程应用。发展高精度、高效率且免自锁的数值方法,成为重要研究课题。

Timoshenko曲梁在有限元分析中常出现剪切自锁现象,这是由于横向位移 w和旋转场 
θ的插值阶次不匹配导致的。为消除锁闭效应,减缩积分和选择性积分技术被广泛采用。
Hughes等人\cite{6}\cite{10}证明了混合有限元方法与减缩-选择性积分位移方法的等价性,
通过减少积分点放松应变约束,适用于线性和非线性问题。基于此方法Cheng等人\cite{5}
采用减缩积分处理剪切能项,结合单点高斯积分和气泡函数空间消除锁定;
Stolarski和Belytschko\cite{9}针对曲单元通过减少高阶应变项的积分点避免膜锁定;
de Melo等人\cite{7}基于减缩积分策略开发Mindlin梁单元消除膜和剪切锁定;Bouclier等人\cite{4}
在等几何分析中应用选择性减缩积分,减少膜和剪切应变能的积分点以放松约束;Zhang等人\cite{3}
指出选择性简化积分作为替代方案具有简单易实现的优势;Choi等人[8]则通过选择性降阶基函数减轻锁定。
然而,减缩-选择性积分方法存在显著缺陷:可能引入沙漏模式导致数值不稳定,Bouclier等人\cite{4}
指出此方法会因沙漏模式引发单元不稳定;其次,计算规则在高阶多项式或非均匀连续性情况下复杂度显著增加,
Zhang等人\cite{3}提到选择性减缩积分规则在考虑高阶基函数和非均匀单元间连续性时变得相当复杂,
限制其通用性;此外,减缩积分在某些场景效果较差,如Erwin Hernández和Jesus Vellojin\cite{15}
指出,相比混合约束应变方法,减缩积分方案通过减少约束方程数量缓解锁定的效果较弱。

减缩积分与选择性减缩积分虽能有效缓解剪切自锁,但仍有不足而,
假定应变理论与B-bar投影方法为缓解剪切自锁提供了另一有效途径。
Patton等人\cite{12}提出等几何假定自然应变方法,通过局部应变投影替代兼容应变场,
利用等几何分析的高阶连续性优势自然消除锁定,无需全局矩阵求逆即可同时处理多类锁定现象。
Prathap和Shashirekha[11]以变分正确方式重构假定应变场提升单元精度,而Choi和Lim\cite{14}
基于假定应变场开发的通用曲梁单元具备快速收敛特性。Elguedj等人\cite{20}
的B-bar投影法将应变投影至低阶基函数空间,有效消除体积与剪切锁定,
兼具通用性与数值稳定性;Bouclier等人\cite{4}进一步结合B-bar投影与选择性减缩积分,
实现无需剪切修正因子的任意曲率梁分析。Greco等人[18]采用混合B样条公式配合局部投影消除膜锁定,
Zhang等人\cite{3}则通过全局投影应用于三维曲梁增强鲁棒性。Sun等人\cite{17}
提出的等阶应变投影方法以位移同阶基函数改善应变解精度,兼具无锁定与计算简便优势。
然而,计算成本显著增加,Bouclier等人\cite{4}指出B-bar方法需全局矩阵求逆导致密集刚度矩阵,
降低计算效率,而Sun等人\cite{17}提及低阶投影可能牺牲应变解精度;其次,方法可靠性受采样点选择影响,
Greco等人\cite{18}强调假定自然应变方法中合适采样点选取困难且易引入误差;此外,部分实现过程复杂,
如Patton等人\cite{12}的等几何假定自然应变需基函数变换增加编程难度,Elguedj等人\cite{20}
的B-bar方法在高阶应用中可能需结合其他技术以完全消除锁定。

尽管减缩积分与假定应变场技术能缓解剪切自锁,但其人为修正可能导致能量不守恒或边界条件失真。
因此,学者转向混合方法通过独立近似应力场与位移场从变分原理层面消除剪切自锁现象。
Saleeb和Chang\cite{24}开发混合-混合公式,独立插值位移场和应力场并通过静态凝聚处理内部参数,
避免膜-剪切锁定,实现网格快速收敛与准确应力预测。Afsin Saritas\cite{25}
基于Hu-Washizu三场变分原理独立插值位移、应力和应变场,通过力插值函数满足平衡方程,
从根本上消除膜锁定和剪切锁定,适用于非线性材料。L. Beirão da Veiga等人\cite{21}
采用等几何配置方法结合混合公式,引入剪切应变作为独立变量并利用B样条基函数的高阶连续性避免剪切锁定,
适用于薄壁结构。Hale和Baiz\cite{44}利用混合变分原理与最大熵基函数消除剪切锁定,提升稳定性。
Bouclier等人\cite{19}结合混合公式和B投影方法独立插值应力场,有效消除薄壁结构锁定。L. Greco等人\cite{18}
采用混合B样条公式通过局部投影和样条重建技术优化应变场,消除膜锁定并提高计算效率。Baier-Saip等人\cite{23}
指出混合插值法通过独立近似场消除剪切锁定,性能优于纯位移法。然而,因需处理多个独立场和全局矩阵求逆,
导致刚度矩阵稠密化并增加计算负担;方法实现复杂,需谨慎选择离散空间以满足稳定性准则,
Baier-Saip等人\cite{23}强调必须满足Ladyzhenskaya-Babuska-Brezzy稳定性条件,否则可能出现伪模式,
而Afsin Saritas\cite{25}的混合公式因需精确应力插值增加编程难度;此外,部分方法可能引入数值振荡或收敛问题,
Hale和Baiz\cite{44}指出混合变分公式需严格处理基函数特性以避免精度损失。

混合方法虽能从理论上消除锁定,但其计算成本高且实现复杂。为平衡精度与效率,
学者提出高阶插值与混合插值法作为替代方案。Likang Li\cite{31}通过p版本和h-p版本有限元方法增加多项式次数p,
理论严谨且能彻底解决剪切锁定;Koziey和Mirza\cite{28}采用三次位移和二次旋转多项式近似确保一致性以避免虚假剪切应变;
Dawe\cite{29}使用五次多项式模型提升薄结构精度;Tai和Chan\cite{38}基于Legendre多项式引入分层高阶形函数增强精度并避免锁定。
在混合插值法方面,Saleeb和Chang\cite{24}基于Hellinger-Reissner原理独立插值应力场以松弛约束;
Saritas\cite{25}利用Hu-Washizu三场变分原理独立插值位移、应力和应变场,从根本上消除膜和剪切锁定;
Bathe和Dvorkin\cite{34}则对剪切应变单独插值并结合Mindlin板理论避免锁定。形函数方法中,
Caillerie等[39]使用三次和二次多项式形函数引入内部自由度并通过静态凝聚避免剪切锁定;
Wang等\cite{35}则引入内部节点和Hermitian/Lagrangian插值函数以增强灵活性。然而,
这些方法均存在局限:高阶插值需足够大的p值才能完全消除锁定,计算成本高且不适用于实时应用\cite{31}
;混合插值法实现复杂且计算昂贵;形函数方法构造依赖特定问题而缺乏通用性,如Tai和Chan\cite{38}需内部自由度导致实现繁琐。

高阶与混合插值法仍受限于计算复杂度与通用性不足。因此,无网格方法因其灵活的离散特性成为新兴研究方向。
Wang和Chen\cite{43}基于MLS/RK近似和稳定共轭节点积分提出无网格曲梁公式,通过曲率平滑技术消除剪切和膜锁定,
可精确再现纯弯曲模式而无寄生变形;Donning和Liu\cite{41}采用基数样条函数作为形函数,
结合位移基Galerkin方法在粗网格下实现高精度,并通过插值层面点对点消除锁定;Baier-Saip等人\cite{42}
基于混合变分公式,融合最大熵基函数与旋转Raviart-Thomas-Nédélec单元,可直接施加狄利克雷边界条件并避免剪切锁定;
Xiao和McCarthy\cite{47}利用MLPG方法结合无锁公式,通过改变控制方程变量消除锁定,无需网格离散;Li等人\cite{45}
采用对称SPH方法改进一致性误差,适用于功能梯度材料但未直接解决剪切锁定;Huang等人[46]基于CSPM法自动满足自由边界条件,
适用于大变形但同样未明确针对锁定问题;Atluri和Zhu\cite{48}提出的MLPG方法通过局部弱形式和移动最小二乘近似处理非线性问题。
然而,无网格方法存在本质边界条件处理困难,Donning和Liu\cite{41}指出其框架中满足本质边界条件仍是重大挑战;
低阶近似中锁定消除更困难,Xiao和McCarthy\cite{47}提到因近似扩散特性导致低阶时锁定缓解效果有限;MLPG方法需局部积分域可能增加计算成本,
且Atluri等人\cite{48}提及该方法存在收敛一致性难以保证的问题。

尽管现有研究已发展出减缩积分、假定应变场、混合方法、高阶插值及无网格方法等多种技术以抑制Timoshenko曲梁分析中的剪切自锁现象,
这些方法仍存在诸多未解决的根本问题,如数值稳定性与计算效率难以兼顾、实现复杂度高、通用性受限,
以及缺乏对自锁机理的统一理论指导。为此,本文将以Timoshenko曲梁为研究对象,
重点探讨其混合离散分析中消除剪切自锁的“最优约束比例”问题,旨在通过理论分析与数值实验明确不同离散策略下约束条件与自锁行为之间的内在关联,
从而填补现有方法在理论指导层面的空白。在此基础上,本文将进一步建立一种“自锁约束比例可控的有限元与无网格混合离散分析方法”,
通过有机结合有限元的结构化离散优势与无网格法的灵活近似特性,实现约束比例的可控调节,在保证计算精度的同时有效避免自锁现象,
为工程实践提供一种兼具鲁棒性与适应性的新型数值解决方案。

\section{研究特色及创新之处}

1.Timoshenko曲梁混合离散分析中消除剪切自锁问题的最优约束比例。
2.自锁约束比例可控的有限元无网格混合离散分析方法。

\section{研究内容与方法}

\subsection{研究内容}

本文将针对Timoshenko曲梁剪切自锁问题,研究位移场和约束场的自由度数量比与LBB稳定性条件影响机理,
并利用最优约束比和无网格法的离散便利性,从而建立Timoshenko曲梁最优约束比的有限元无网格混合分析方法。具体内容如下:


\begin{itemize}
    \item 建立Timoshenko曲梁有限元无网格混合离散理论框架,设计横向位移 w和旋转场 θ采用二次再生核无网格形函数离散;
    约束场采用有限元离散,并研究可任意调整自由度数量的无网格法节点布置方案,从而优化形函数影响域大小。
    \item 将Timoshenko曲梁最优约束比例引入有限元无网格混合离散分析方法中,
    从而建立最优约束比的免自锁有限元无网格混合离散分析方法。为复杂曲梁结构的锁紧问题提供了新的解决路径。
\end{itemize}

\section{研究计划与预期成果}
预期将取得以下成果:
1.建立Timoshenko曲梁剪切自锁与LBB稳定性条件的关系,确定剪切自锁的最优约束比例。
2.建立曲梁最优约束比的有限元无网格混合离散分析方法,在保证计算精度的同时消除自锁现象。


\subsection{研究计划}

详细的研究时间安排和阶段性目标。

\begin{figure}[H]
\centering
\includegraphics[width=0.8\linewidth]{1.png} 
\caption{研究计划时间表}
\end{figure}

% ====================================================================
% 注意:请在 references.bib 文件中添加对应的参考文献条目
% ====================================================================